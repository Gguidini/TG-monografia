% REFERENCIAL TEÓRICO
% ------------------- 
% Levantamento da literatura,
% conceitos importantes, 
% trabalhos relacionados
\label{chapter:referencial}

O referencial teórico apresenta conceitos importantes para o projeto e trabalhos relacionados encontrados na literatura. A Seção \ref{sec:outros_simuladores} descreve um breve levantamento feito de alguns simuladores bem estabelecidos na literatura. Esses projetos foram usados de inspiração para a criação do HMR Sim. A Seção \ref{sec:ECS} apresenta a \textit{design pattern} ECS (\textit{Entity-Compoenent-System}), analisada por ser bastante utilizada na criação de jogos, pela complexidade dos jogos de videogame atuais e como estes são de certa forma simuladores. Esta arquitetura foi utilizada no projeto do HMR Sim pelas suas vantagens. Finalmente a Seção \ref{sec:simulation_techniques} comenta sobre duas técnicas de simulação encontradas nos simuladores que foram levantados e na literatura.

\section{Simuladores na Literatura}
\label{sec:outros_simuladores}

O levantamento da literatura iniciou com um breve levantamento de simuladores já estabelecidos para sistemas robóticos. Foram selecionados os simuladores Gazebo \cite{koenig2004gazebo}, Simbad \cite{hugues2006simbad}, CoppeliaSim \cite{rohmer2013coopeliasim}, MORSE \cite{echeverria2011morse} e Dragonfly \cite{maia2019dragonfly}. Uma descrição breve de como os robôs são definidos em cada simulador é incluída, junto com um link para a documentação oficial do projeto.

Cada um desses simuladores foi criado com propostas diferentes, desde simulação precisa das partes que compõem um robô e sua interação com o ambiente (Gazebo, CoppeliaSim, Morse), até simulações de mais alto nível focando principalmente no comportamento dos robôs (Simbad, Dragonfly). Simulações multi-robô são suportadas por simuladores atuais, mas geralmente em menor número - devido ao alto uso de recursos computacionais necessários para simular cada robô (i. e. experimentos feitos com Gazebo mostraram que o simulador tem dificuldades ao simular mais de 10 robôs \cite{noori20173d}) - ou são muito específicos quando conseguem simular mais robôs (i.e. Dragonfly supostamente é capaz de simular até 400 entidades, mas só consegue simular drones \cite{maia2019dragonfly}).

\textbf{Gazebo.} Robôs são definidos em modelos, que seguem uma estrutura de arquivos definida. O robô em si é descrito em arquivos .sdf. Cada modelo é definido através de uma série de <links> que definem as partes do modelo. Sensores ou outros componentes - outros modelos - podem ser ligados através de <joints>. Modelos podem ter plug-ins com funcionalidade extra. O projeto é open source e pode acessado no link \url{http://gazebosim.org}. 

\textbf{CoppeliaSim.} Modelos são definidos como uma seleção de scene objects (e.g. joints, shapes, sensors, cameras, paths, etc). Existem muitas maneiras de controlar uma simulação, dentre elas destaca-se embedded scripts. Esse scripts podem ser definidos como parte de um scene object e executam alguma funcionalidade relacionada à este objeto. CoppeliaSim (antigamente V-Rep) é uma solução comercial da empresa Coppelia Robotics, disponível no link \url{https://www.coppeliarobotics.com/features}.

\textbf{MORSE.} Robôs são plataformas que definem o formato e certas propriedades, como área de colisão massa, etc. É nessas plataformas que sensores e atuadores são montados. Estes são fornecidos pelo simulador MORSE para serem adicionados à robôs. Apenas sensores e atuadores interagem com o mundo real com alguma funcionalidade. Os sensores e atuadores são fornecidos em diversos níveis de realismo, permitindo maior ou menor grau de abstração na simulação. MORSE é uma solução open source, baseado no software de modelagem Blender. Infelizmente o projeto se encontra abandonado desde 2020, mas ainda está disponível no link \url{https://morse-simulator.github.io}.

\textbf{Simbad.} Robôs são definidos em classes Java que estendem a classe \texttt{Agent}. Sensores são adicionados como atributos da classe. Status e movimentação do robô é alcançado através de APIs próprias. Robôs implementam as funções \texttt{initBehavior()} e \texttt{performBehavior()}, que definem o que acontece com o robô ao ser criado e o comportamento dele em cada loop de simulação. Um projeto open source disponível no link \url{http://simbad.sourceforge.net}.

\textbf{Dragonfly.} Drones possuem uma classe de controle - \texttt{DroneKeyboardController} ou \texttt{DroneAutomaticController}, respectivamente para ser controlado pelo usuário ou automaticamente. E classes que definem seus comportamentos, através de modelos. Além disso outras configurações, como nível de bateria, consumo por bloco, alvo e os wrappers (para fornecer comportamento adaptativo)  podem ser alterados pela interface gráfica, para cada drone na cena. Esse simulador está limitado à simuações de drones. Também open source, disponível em \url{https://github.com/DragonflyDrone/Dragonfly}.

Cada simulador possui características arquiteturais e objetivos próprios que foram analisados e comparados, fornecendo um arcabouço de técnicas que podem ser utilizadas (ver Tabela \ref{table:simulators_comparison}). Pontos relevantes que foram investigados sobre os simuladores incluem: 

\begin{itemize}
    \item \textbf{Nível de Abstração.} Pode ser \emph{baixo} indicando grande detalhamento dos componentes que compõe o robô e suas características físicas; \emph{médio} indicando necessidade de detalhamento dos movimentos individuais dos componentes que formam um robô; \emph{alto} indicando abstração dos componentes do robô.
    \item \textbf{Número de robôs} que o simulador é capaz de simular num tempo razoável
    \item Se o simulador é \textbf{genérico} ou não, ou seja, se existe restrição no tipo de robôs que o simulador é capaz de simular.
    \item \textbf{Arquitetura} utilizada na representação do robô (i.e. um robô é uma classe que deve ser implementada, ou um arquivo XML, etc)
    \item \textbf{Tipo de simulação} indica qual a técnica de simulação usada, em passos ou de eventos discretos (ver Seção \ref{sec:simulation_techniques})
\end{itemize}

\begin{table}[]
    \hspace{-0.8cm}
    \begin{tabular}{rccccc}

    \multirow{2}{25mm}{\textbf{Simulador}}      &
    \textbf{Nível de}                           &
    \multirow{2}{25mm}{\textbf{Nº de robôs}}    &
    \multirow{2}{25mm}{\textbf{Genérico}}       &
    \multirow{2}{25mm}{\textbf{Arquitetura}}    &
    \textbf{Tipo de}                            \\
    
                                    &
    \textbf{Abstração\Bstrut}       &
                                    &
                                    &
                                    &
    \textbf{Simulação\Bstrut}       \\

    \midrule

    {\textbf{Gazebo}\Tstrut} &
        {Baixo} &
        {\textless 20} &
        {SIM} &
        {Declarativa} &
        {Passos/DES} \\
        
    {\textbf{CoppeliaSim}\Tstrut} &
        {Baixo} &
        {\textless 20} &
        {SIM} &
        {Declarativa} &
        {Passos/DES} \\
        
    {\textbf{Simbad}\Tstrut} &
        {Médio} &
        {\textless 10} &
        {SIM} &
        {OOP} &
        {Passos} \\
        
    {\textbf{MORSE}\Tstrut} &
        {Médio/Alto} &
        {20 - 100} &
        {SIM} &
        {OOP/Declarativa} &
        {Passos} \\
        
    {\textbf{Dragonfly}\Tstrut} &
        {Alto} &
        {400} &
        {NÃO} &
        {MVC/AOP} &
        {Passos} \\
        
%     % {\textbf{HMR Sim*}} &
%     %     {\begin{tabular}[c]{@{}c@{}}Médio /\\ Alto\end{tabular}} &
%     %     {50 - 500} &
%     %     {SIM} &
%     %     {\begin{tabular}[c]{@{}c@{}}ECS/\\ Declarativa\end{tabular}} &
%     %     {\begin{tabular}[c]{@{}c@{}}Passos /\\ DES\end{tabular}}
    \end{tabular}
    \caption{Comparação resumida das ferramentas com suporte de simulação multi-robôs.}
    \label{table:simulators_comparison}
\end{table}

\section{Entity-Component-System}
\label{sec:ECS}

\textit{Entity-Component-System} (ECS) é um padrão de desenho (\textit{design pattern}) de software amplamente utilizada em jogos, tipicamente em sistemas interativos em tempo-real (e.g. jogos do tipo MMO, \textit{Massive Multiplayer Online}) \cite{wiebusch2015decoupling}. Nesse padrão, objetos da simulação são transformados em \textit{entidades}. Cada entidade nada mais é que uma coleção de \textit{componentes}. Um componente, por sua vez, armazena dados, mas tipicamente não implementa nenhuma lógica.

A lógica da simulação está nos \textit{sistemas}, que modificam os dados de componentes de acordo com seu objetivo. Cada sistema age de maneira independente de outros sistemas sobre um conjunto de componentes que lhe interessa, ou seja, se uma entidade possui esse conjunto de componentes, então ela será afetada pelo sistema durante a simulação. O estado da simulação é o conjunto de estados de todos os componentes de todas as entidades presentes na simualção. Ele é alterado apenas pelos sistemas, cada um alterando uma pequena parte desse estado global.

Essa organização permite grande modularização e separação de lógica entre as difrentes partes do sistema. Cada sistema (ou conjunto de sistemas) e seu conjunto de componentes associados pode ser adicionado ou remivdo do simulador conforme necessário. Por exemplo, um sistema comunicação entre diferentes robôs pode ser implementado como um componente que guarde uma fila de mensagens e pode ser adicionado à cada robô, associado à dois sistemas: um sistema que faça a entrega das mensagens de um robô para o outro, e outro sistema que processa as mensagens de cada robô. Note que se o processamento não for adequado à uma simulação, basta trocar aquele sistema por outro que seja adequado. Além disso, se alguma simulação não faz uso desse sistema de mensagens, basta removê-lo do simulador completamente, deixando a simulação mais leve.

Uma outra vantagem de utilizar o padrão ECS é a flexibilidade de adicionar ou remover capacidades das entidades durante a execução da simulação. Como cada entidade é simplesmente uma coleção de componentes, é possível associar certas capacidades dos robôs (e.g. sensores, atuadores) à presença ou ausência de certos componentes naquela entidade. Por exemplo, dada a existência de um componente \texttt{camera} e um sistema associado que simule a captura de imagens, qualquer entidade que possue esse componente vai possuir a capacidade de coletar imagens via componente camera. Além disso, al simular falhas catastróficas em componentes, basta remover o componente da entidade sendo analisada.

Apesar dessas vantagens, como apontado por Wiebush \cite{wiebusch2015decoupling}, o uso de ECS pode trazer complicações de compatibilidade entre sistemas desenvolvidos de maneira indepentente, como uso de componentes incompatíveis, e dificuldade em conhecer qual sistema é responsável por determinada funcionalidade e como utilizá-la. detalhes de como esses problemas foram sentidos durante o desenvolvimento do projeto e medidas tomadas para mitigá-los são discutidas no capítulo \ref{chapter:hmr_sim}.

Foi utilizada a biblioteca \texttt{esper} para suporte do padrão ECS. Esper é uma biblioteca de ECS leve com foco em performance, escrita na linguagem Python por Benjamin Moran \cite{esper}. Ela cria uma classe \texttt{World} que mantém uma lista de entidades e de todos os componentes para cada entidade. Um componente pode ser qualquer estrutura em Python, no caso do projeto foram usadas classes (i.e. \texttt{class}). É possível ainda adicionar sistemas à classe \texttt{World}, que são implementados como funções, convencionalmente chamadas \texttt{process}. Alguns dos sistemas do projeto são adicionados ao \texttt{World}.

\section{Técnicas de Simulação}
\label{sec:simulation_techniques}

Uma técnica de simulação bem estabelecida é a de tempo discreto com intervalo de incremento fixo \cite{belanger2010aboutsimulation}. Nesse modelo, o estado de um sistema no tempo $t_{i+1}$ é uma função do estado do sistema no tempo $t_i$. Cada variável que compõe o estado do sistema é uma função de variáveis e estados até o momento anterior. O incremento de tempo da simulação entre $t_i$ e $t_{i+1}$ é sempre o mesmo, e pré-definido.

Se o tempo $t_{calc}$ necessário para computar o estado $t_{i+1}$ do sistema a partir do estado $t_i$ é menor do que o tempo do incremento $t_{incr}$, então a simulação será computada mais rápido do que o tempo do relógio (e.g. o tempo real); da mesma forma, se $t_{calc} > t_{incr}$, então a simulação é computada mais devagar do que o tempo do relógio. Essas situações são conhecidas como simulação \textit{offline} \cite{belanger2010aboutsimulation}, porque não há sincronia entre o tempo da simulação e o tempo do relógio. Essa é uma situação aceitável para este projeto, onde o objetivo é obter a simulação desejada no menor tempo possível.

Essa técnica de simulação é indicada para simular sistemas que mudam constantemente, como por exemplo a temperatura de um ambiente ao longo do tempo, ou um sinal recebido por um sensor que trabalha a uma frequência conhecida. No entanto o  "relógio" da simulação é sincronizado, e todas as funções do estado são processadas a cada incremento de tempo, o que pode levar a cálculos desnecessários. Por exemplo, em uma simulação que involva uma função que altera temperatura de uma sala a cada 200ms, e um sensor que registra a temperatura da mesma sala com leituras a cada 100ms, a função que altera temperatura deve ser executada em todos os incrementos de tempo, que devem ser no máximo 100ms para suportar a leitura do sensor. Nesse cenário, metade das chamadas à função de alterar temperatura não afeta o estado do sistema, mas ainda tem que ser processadas.

Outra técnica de simulação é por eventos discretos (DES, \textit{Discrete Event Simulation}) \cite{matloff2008desintro}. Nesse modelo uma fila de eventos é processado um por vez, e cada estado $s_{i+1}$ é o resultado de processar o evento no topo da fila sobre o estado $s_i$. Um evento $e$ possui um tempo $t$ e uma função $f$ que altera o estado, e potencialmente cria outros eventos, que serão adicionados à fila. A fila de eventos é uma fila de prioridades ordenada pelo tempo $t$ de cada evento, sendo que o tempo de cada novo evento gerado pode ser igual ou maior que o tempo do evento que o gerou (nunca menor, porque não se pode alterar o passado da simulação). O tempo da simulação corresponde ao tempo $t$ do evento atual sendo processado, e como os eventos são ordenados pelo tempo, ele só será incrementado quando todos os eventos naquele tempo foram processados.

Diferentemente da simulação com intervalo de incremento fixo, onde as mesmas funções são executadas em intervalos conhecidos de tempo, na simulação do tipo DES funções diferentes alteram o estado da simulação, e o tempo da simulação no estado $s_i+1$ não depende apenas do estado $s$, mas também do evento sendo processado. Esse novo tempo pode não crescer de maneira uniforme ao longo da simulação. Essa técnica é adequada para simular sistemas que mudem de maneira infrequênte ao longo do tempo, por exemplo o inventário de um armazém \cite{belanger2010aboutsimulation}, ou a operação de robôs de serviço dentro do armazém. 

A simulação de incremento fixo de tempo pode ser implementada utilizando a técnica de eventos discretos, desde que os eventos sejam criados com tempos que possuam um intervalo constante. Uma outra característica interessante que pode ser alcançada com eventos discretos é separar a função em subsistemas que são executados de maneira independente e assíncrona. Retomando o exemplo da sala que muda de temperatura e possui o sensor, cada evento de leitura do sensor pode criar o próximo evento de leitura para o tempo $t + 100ms$; de forma similar cada evento de mudança de temperatura cria um novo evento de mudança para o tempo $t + 200ms$. Dessa forma, evita-se o problema de funções de alteração do estado da simulação tendo que ser executadas antes da hora.

O simulador utiliza a técnica de simulação de eventos discretos, através da biblioteca \texttt{simpy} \cite{simpy}. Esse framework de simulação DES é baseado em processos e faz todo o gerenciamento dos eventos e sua execução. A simulação acontece dentro de um ambiente, onde diversos processos interagem entre si e com o ambiente através de eventos. QUalquer função geradora em Python pode ser um process no simpy.

Esse framework também tem suporte para recursos (\textit{Resources}), que são compartilhados entre os processos. Recursos podem simular desde recursos a serem disputados (i.e. uma impressora, uma estação de carga) até recursos que são armazenados em contâiners (i.e. 10L de aguá de um reservatório com capacidade para 10000L). Recursos podem ainda ser preemptivos, ou filtrados de algum contâiner. Essa última capacidade foi bastante utilizada para a comunicação entre sistemas do simulador, como será discutido no Capítulo \ref{3_HMR_sim}.
