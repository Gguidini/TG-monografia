% INTRODUÇÃO
% ----------
% Apresentar contexto geral do projeto - sistemas multi-robôs heterogêneos
% Apresentar as dificuldades no desenvolvimento desses sistemas
% Motivação do projeto
% Objetivos gerais
% Objetivos específicos
% Layout geral
\label{chapter:intro}

Sistemas Multi-Robôs (SMR) são sistemas que consistem em mais de um agente robótico. Por algumas décadas esses sistemas foram utilizados em diversos contextos para cumprir diversas tarefas, especialmente em ambientes dinâmicos. SMRs atuam em um espaço ciber físico (i.e. parte do “mundo real”), logo seus agentes estão propensos à mudanças provenientes tanto de outros agentes do sistema como do ambiente em que estão inseridos \cite{iocchi2000reactivity}. Para aumentar a adaptabilidade do SMR, pode-se projetá-lo como um sistema auto-adaptativo, tornando-o capaz de responder à mudanças no ambiente, de maneira a continuar cumprindo objetivos e respeitando os limites impostos ao sistema \cite{sykes2010autonomous}.

Os agentes desses sistemas (i.e. robôs) existem no mundo físico e interagem com ele e entre si de maneiras mais complexas do que agentes de outros sistemas (e.g. computadores, bancos de dados, etc) \cite{cao1997cooperative}. Isso traz desafios para o desenvolvimento desse tipo de sistema, principalmente a preparação de experimentos com vários robôs \cite{noori20173d}. Essa dificuldade pode ser superada com o uso de simuladores. Simuladores podem ser empregados tanto para testar a segurança, eficiência e robustez do sistema, quanto para prototipação de SMRs e robôs \cite{noori20173d}, \cite{pinciroli2012argos}. Outras vantagens de simuladores incluem: (1) menor custo de tempo e recursos para preparação e execução do experimento; (2) ambientes simulados podem ser mais ricos, complexos e seguros que ambientes reais ou em laboratório; (3) é possível testar hardware que não está disponível \cite{pinciroli2012argos}, \cite{echeverria2011morse}.  

Diversos simuladores para SMRs existem na literatura, por exemplo Gazebo \cite{koenig2004gazebo}, Simbad \cite{hugues2006simbad}, CoppeliaSim \cite{rohmer2013coopeliasim}, MORSE \cite{echeverria2011morse} e Dragonfly \cite{maia2019dragonfly}, entre outros. Cada um desses simuladores foi criado com propostas diferentes, desde simulação precisa das partes que compõem um robô e sua interação com o ambiente (Gazebo, CoppeliaSim, Morse), até simulações de mais alto nível focando principalmente no comportamento dos robôs (Simbad, Dragonfly). Simulações multi-robô são suportadas por simuladores atuais, mas geralmente em menor número - devido ao alto uso de recursos computacionais necessários para simular cada robô (i. e. experimentos feitos com Gazebo mostraram que o simulador tem dificuldades ao simular mais de 10 robôs \cite{noori20173d}) - ou são muito específicos quando conseguem simular mais robôs (i.e. Dragonfly supostamente é capaz de simular até 400 entidades, mas está restrito à simulação de drones \cite{maia2019dragonfly}).

O Laboratório de Engenharia de Software (LES) da Universidade de Brasília (UnB) conduz pesquisas na área de sistemas multi-agentes, incluindo sistemas multi-robôs. Entre os simuladores empregados nas pesquisas do LES, encontram-se Gazebo e MORSE, porém tem sido relatadas dificuldades com o uso desses simuladores em cenários com times maiores de robôs. Isso se dá pelo alto nível de detalhamento físico das simulações, que exige recursos computacionais consideráveis. Quando o objetivo da pesquisa é mais voltado para os algoritmos que coordenam os diferentes agentes do sistema, ese nível alto de detalhamento é desnecessário, mas aumenta consideravelmente o tempo de cada experimento.

Nesse cenário, a proposta deste projeto é fornecer uma ferramenta direcionada para simulação de sistemas multi-robôs auto-adaptativo com baixo nível de detalhamento físico. Esta ferramenta será usada na avaliação e comparar algoritmos de distribuição de tarefas entre agentes de um SMR. Também pode ser usado para prototipação das características de cada agente do SMR, bem como validação de requisitos do time de robôs de algum sistema.

O capítulo \ref{chapter:referencial} comenta conceitos que foram relevantes na construção do simulador, e comenta um pouco mais sobre os outros simuladores que foram pesquisados na literatura. O capítulo \ref{3_HMR_sim} apresenta o simulador que foi criado nesse projeto, HMR Sim, bem como sua arquitetura, decisões de projeto importantes e características principais. O capítulo \ref{chapter:results} detalha alguns resultados obtidos com o HMR Sim, bem como conclusões e trabalhos futuros.


